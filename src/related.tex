\section{Related Work}
\label{section: related work}

Outside of \cite{ding2024defeasiblereasoningconcepts} we are not aware of any direct attempts to introduce non-monotonic reasoning of any kind to FCA. In \cite{ding2024defeasiblereasoningconcepts}, the authors attempt to introduce KLM-style defeasible reasoning on the sub/super concept relation in FCA. Specifically, where in classical FCA $C_1 \vdash C_2$ indicates that ``all the objects in concept $C_1$ are in concept $C_2$'', their aim is to introduce a non-monotonic operation $C_1 \twiddle C_2$ meaning ``the \textit{typical} objects in $C_1$ are in $C_2$''. They restrict their attention to cumulative reasoning, and thus avoid issues around disjunction and negation.

In addition, there is some work which runs parallel to broad ideas around ``typical concepts''. \cite{kent1996rough} introduces \textit{Rough Concept Analysis}, a merging of rough set theory and FCA that uses equivalence classes on objects to define upper and lower-bound approximations of concepts. \cite{yao2016rough} investigates an expansion which enables ``rough concepts'' to be defined not only by objects.