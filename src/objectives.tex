\section{Research Questions \& Objectives}
\label{section: research objectives}

We now develop a more thorough framing of the problems discussed in \cref{section: motivation}, detailing the current view on what the correct questions are, and how we undertake finding their solutions.




% conceptual structuring of data, ontology engineering, software mining



% The aim of this work is to introduce KLM style non-monotonicity to the attribute logic underpinning FCA. Doing so creates two principle areas of interest. The first, and most obvious, concerns the notion of a \textit{non-monotonic implication} in FCA. Secondly, a corollary of introducing a ranking to a context is that it a provides expressivity to develop \textit{typical concepts}.

% Concerning non-monotonic implications, we begin our work by finding a translation from the weaker \textit{system P} to FCA. This system relies on being able to construct an ordering over the valuations of a knowledge base. We find that assuming a partial ordering over the set of objects in a formal context—implicitly providing a way to compare object intents—introduces a suitable structure for a semantic definition of preferential implications. \cref{definition: modelling an implication} can be altered to define a defeasible implication which is valid in a context iff the minimal objects in $A'$ are a subset of $B'$.

