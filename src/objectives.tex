\section{Research Questions \& Objectives}
\label{section: research objectives}

In light of \cref{section: motivation}'s argument for introducing non-monotonic reasoning to FCA, we now develop a more thorough framing of the problem. Ultimately, the aim of this work is to develop a holistic notion of KLM-style non-monotonic reasoning in FCA. That is, a semantics for rational consequence that gives rise to both defeasible implications and ``rational concepts''. The following details the pertinent questions, and how we intend to approach them.

The starting point is to begin developing a syntax and semantics for preferential reasoning in an FCA setting. More specifically, the aim is to develop semantics for implications which give rise to a rational consequence relation. The current approach finds an analogue between preference over worlds (from \cite{shoham1987reasoning,kraus1990nonmonotonic}) and a preference over the objects in a formal context. Progressing from this idea to a semantics is quite intuitive. This brings about immediate questions regarding the \textit{Or} and \textit{Rational Monotonicity} postulates: typically, the attribute logic of FCA does not have the expressivity for negation or disjunction, which are central to these postulates. To this end, it needs to be investigated how this expressivity can be introduced to FCA, and what their ramifications may be.

To make this explicit, the aim is to enable a non-monotonic implication $A \rightsquigarrow B$ between attribute(s). This implication would be valid in a formal context if it is respected by the most typical (i.e., preferred, best) object intents which are supersets of $A$. It would then be of interest to investigate how we might naturally induce an ordering over objects. An idea to be explored further is to supplement a formal context with a defeasible knowledge base, $\mathcal{K}$, which acts as an authoritative view over the domain. The statements in $\mathcal{K}$ are then ranked, which in turn provides a way to assign each object a rank.

Recalling the discussion in \cref{section: introduction}, one of the uses for FCA was mining (classical) implications as a method of discovering relationships between attribute(s). We would aim to introduce an parallel notion for mining defeasible implications from a formal context. This suggests developing a notion of defeasible entailment from a formal context; specifically, rational closure in FCA. The intuition for what this means is quite clear: given a ranking of objects, can we find a closed set of defeasible implications which are valid in the context. While the rational closure already has a clear definition, part of this work would involve translation from the setting of propositional logic to the attribute logic of FCA. Alongside this task, we intend to evaluate complexity of the algorithm in its new setting.

Echoing the final points of \cref{subsection: formal concept analysis}, in certain cases it can be beneficial to use the set of (classical) implications of a formal context to derive the set of concept intents. We are interested in exploring this in relation to the rational closure of a formal context, and whether it leads to, or if we can otherwise create, a notion of a \textit{rational concept}. Rational concepts are intended to convey the sense of a ``typical'' version of some concept. We return to the example of mammals from \cref{section: motivation}: the rational concept for mammal might describe the attributes we (rationally) associate mammals with, such as giving birth to live young.

It is perhaps a good idea to clarify that the intention is not to create concepts the necessarily align with human psychology or intuition (see \cite{belohlavek2012basic} for this kind of approach). Rather, the concepts should coincide with the typicality relation encoded by the ranking on objects and subsequent rational closure.

The existence of rational concepts raises another point of interest: if a similar ordering of sub/super concept(s) can be induced over rational concepts—ideally, this should be the case—then investigating the resulting structure would be of great interest.

In FCA, the set of all implications valid in a formal context contains many redundant implications (i.e., if $A \rightarrow C$ then via monotonicity $A \cup B \rightarrow C$). This has led to the development of algorithms to find the \textit{canonical basis} of a formal context. That is, the smallest set of implications from which all other implications follow \cite{ganter2016conceptual}. Although not central to the scope of this work, and perhaps not a question we will attempt to solve, attempting to find an analogous basis for defeasible implications would be an endeavour which may provide some utility to the larger KLM community.

