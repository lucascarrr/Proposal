\section{Introduction}
\label{section: introduction}

Formal concept analysis (FCA) provides a framework, grounded in lattice theory, for mathematically reasoning about \textit{formal concepts} and their hierarchies \cite{ganter1999formal,ganter2016conceptual,rudolph2007relational}. The matter of \textit{concepts} has largely been of Philosophical concern: the notion of a concept as the dualism between \textit{intension} and \textit{extension} has foundations in Aristotle's \textit{Organon} and, much later on, in the \textit{Logic of Port-Royal} \cite{rudolph2007relational,castonguay2012meaning}. In this view, the extension of a concept contains to those ``things'' that one might refer to as instances of the concept. Dually, intension describes the meaning, or \textit{sense}, of a concept.

Formal concept analysis---which adopts this view of concepts---introduces a \textit{formal context} as the structure of data. This is a triple consisting of a finite set of objects $G$, attributes $M$, and a binary relation, $I \subseteq G\times M$, which indicates that a particular object has a respective property \cite{ganter1999formal,ganter2016conceptual}. A \textit{formal concept} is a pair, made up of the formal concept extension and intension, respectively. The set of all concepts, when ordered by the \textit{sub/super-concept} relation, form the \textit{concept lattice} used for analysis.


Another important topic in FCA is the discovery of \textit{implications} that pertain to, or are \textit{respected} by, a context \cite{rudolph2007relational,ganter1999formal}. Implications are used to express correspondence between (sets) of attributes. The notion of a context \textit{respecting} an attribute implication is analogous to that of \textit{entailment} in classical logic \cite{ganter2016conceptual}. As such, \textit{respecting} describes a monotonic notion of consequence.

Discussion and work on non-monotonic propositional, first-order, and description logic is a well established topic in artificial intelligence, \cite{ferguson2003monotonicity,giordano2015semantic,kraus1990nonmonotonic,lehmann1994what,shoham1987nonmonotonic}. However, there does not appear to be any effort to introduce this expressivity to the attribute logic of FCA.

% That is not to say that increasing the expressivity of FCA's attribute logic would not be beneficial: association rules are a fairly blunt way of gaining some non-monotonicty in FCA. However, they acquire meaning through majority rule - where we would like something more precise.

% Applications of FCA: text mining, web mining, and ontology engineering
