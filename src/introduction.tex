\section{Introduction}
\label{section: introduction}

Formal concept analysis (FCA) provides a framework, grounded in lattice theory, for mathematically reasoning about \textit{formal concepts} and their hierarchies \cite{ganter1999formal,ganter2016conceptual,rudolph2007relational}. The matter of \textit{concepts} has largely been a Philosophical concern; and the notion of a concept as the dualism between \textit{intension} and \textit{extension} has foundations in Aristotle's \textit{Organon} and, much later on, in the \textit{Logic of Port-Royal} \cite{rudolph2007relational,castonguay2012meaning}. In this view, the extension of a concept refers to those ``things'' which one might point to as instances of the concept. Dually, intension describes the meaning, or sense, of the concept.

Formal concept analysis, which adopts this view of concepts, introduces a \textit{formal context}: a triple consisting of a finite set of objects $G$, and attributes $M$, and a binary relation $I \subseteq G\times H$ indicating that a particular object has a respective property \cite{ganter1999formal,ganter2016conceptual}. \textit{Formal concepts} are then pairs of sets which describe the extension and intension, sourced from the context\footnote{Hereafter, the term 'Formal' will be omitted, and 'Concept' and 'Context' will refer to 'Formal Concept' and 'Formal Context,' respectively, unless otherwise specified.}. The set of all concepts, when ordered by the \textit{sub/super-concept} relation, form the \textit{concept lattice} used for analysis.

Another important topic in FCA is the discovery of \textit{implications} pertaining a context \cite{rudolph2007relational,ganter1999formal}. Implications are used to express correspondencies that exist between (sets) of attributes in a given context.

The use
Applications of FCA: text mining, web mining, and ontology engineering