\section{Introduction}
\label{section: introduction}

Formal Concept Analysis (FCA) provides a lattice theoretic framework for mathematically reasoning about \textit{formal concepts} and their hierarchies \cite{ganter1999formal,ganter2016conceptual}. The matter of \textit{concepts} has largely been of philosophical concern: the view of concepts as a dualism between \textit{intension} and \textit{extension} has its foundations in Aristotle's \textit{Organon} and, much later on, in the \textit{Logic of Port-Royal} \cite{rudolph2007relational,castonguay2012meaning}. Here, the \textit{extension} of a concept as ``those things'' that one would reference as instances of the concept. Dually, the \textit{intension} describes the meaning, or sense, of a concept.

FCA—which adopts this view of concepts—introduces a \textit{formal context} as a data structure from which concepts can be derived. This is a triple consisting of a finite set of objects $G$, attributes $M$, and an incidence relation $I \subseteq G\times M$, which indicates when a particular object has a respective attribute \cite{ganter1999formal,ganter2016conceptual}. A \textit{formal concept} is a pair comprised of an extension and intension. That is, a set of objects and of attributes, respectively. When the set of all concepts belonging to a context is ordered by the \textit{sub/super-concept} relation, a lattice is formed called the \textit{concept lattice}.

Another important topic in FCA is the discovery of \textit{implications} that pertain to, or are \textit{respected} by, a context \cite{rudolph2007relational,ganter1999formal}. Implications are used to express correspondence between (sets of) attributes. The semantics of these implications resemble Tarskian notions of logical consequence, and are accordingly monotonic. Implications hold only if they hold in every object intent (i.e., \textit{valuation}).

The property of monotonicity in the logic underlying FCA means that it is ill-suited to reason about concepts with \textit{exceptional members}, or \textit{defeasible} implications. This work argues that developing the expressivity to do so would be useful for the kinds of problems FCA is used to tackle. Despite this, there have been few attempts to introduce non-monotonic reasoning of any kind to FCA. We attempt this through introducing preferential reasoning, and more specifically the KLM framework to the attribute logic of FCA, developing notions of a non-monotonic conditional, and concepts which behave in accordance to the defined preferential view.

% Discussion and work on non-monotonic propositional, first-order, and description logic is a well established topic in artificial intelligence, \cite{ferguson2003monotonicity,giordano2015semantic,kraus1990nonmonotonic,lehmann1994what,shoham1987nonmonotonic}. However, there does not appear to be any effort to introduce this expressivity to the attribute logic of FCA.

The rest of this proposal is structured as follows: \cref{section: background} is devoted to providing a brief introduction to both FCA (\cref{subsection: formal concept analysis}) and KLM-style non-monotonic reasoning (\cref{subsection: non-monotonic reasoning}). \cref{section: motivation} then presents arguments in favour of introducing non-monotonicity into FCA. In turn, \cref{section: research objectives} clarifies the previous section, explaining the current view of how this might be achieved, as well as identifying particularly challenging obstacles. We discuss related work(s) in \cref{section: related work}, and then provide administrative details concerning this project in \cref{section: ethics} and \cref{section: project plan}.

% That is not to say that increasing the expressivity of FCA's attribute logic would not be beneficial: association rules are a fairly blunt way of gaining some non-monotonicty in FCA. However, they acquire meaning through majority rule - where we would like something more precise.

% Applications of FCA: text mining, web mining, and ontology engineering
