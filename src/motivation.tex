\section{Motivation for Research Area}
\label{section: motivation}

In practice, FCA has been implemented as a framework for information retrieval, program analysis, ontology engineering \cite{poelmans2013formal,haav2004semi,poelmans2012text}. Until very recently \cite{carr2024nonmonotonicextensionsformalconcept,ding2024defeasiblereasoningconcepts} there have been no attempts to lift the expressivity of FCA's attribute logic to a non-monotonic counterpart. To justify non-monotonicity as a desirable property in FCA, toy example is provided.

When introducing younger children to the vertebrates it is often expressed that a defining feature of mammals is that they give birth to live young, and that reptiles lay eggs instead. Mammals and reptiles can be formalised as concepts \texttt{m} and \texttt{r}:
\small
\[
    \begin{aligned}
         & \texttt{m} \coloneq \texttt{\big(\{platypus,\ldots,horse\}, \{warm-blooded,\ldots,live young\}\big)}  \\
         & \texttt{r} \coloneq \texttt{\big(\{snake,\ldots,J-chameleon\}, \{cold-blooded,\ldots,lay eggs\}\big)}
    \end{aligned}
\]
\normalsize
There is, however, a problem with this construction. Namely, Jackson chameleons are reptiles that give birth to live young, and platypodes are mammals that lay eggs, and so they should not belong to these concepts. The classical FCA response to this issue might be that there should be two sub-concepts, \texttt{m$_1$} and \texttt{m$_2$}, of \texttt{m}. Then, \texttt{m$_1$} may specify those mammals which give birth to live young, and \texttt{m$_2$} those that lay eggs.

This argument forces the abandon the kind of reasoning that appears so central to our intuition. We do not think of reptiles as being divided between those that give birth to live young and those that do not; moreover, there are likely several properties which cause similar rifts in conception. A more natural way of reasoning about this matter would be to think of Jackson chameleons as \textit{exceptional} reptiles. \textit{Exceptionality} obviously suggests a counter notion of \textit{typicality}. Then, we might express that typical reptiles lay eggs, while exceptional ones may give birth to live young.

Another important concern is the use implications of a formal context to discover correspondence between attribute(s). As a reminder of \cref{definition: modelling an implication}, classical FCA implications are only valid in a formal context when they are respected by every object. Consequently, a context modelling birds and their attributes might find that, although most objects have wings and fly, there are exceptions (e.g., penguins, ostriches) which prevent \texttt{wings} $\rightarrow$ \texttt{fly} from being valid; and so we remain unable to express the relationship between these attributes. Once again, the classical framework does not handle exceptions well.

A method to describe implications which only partially hold in a formal context is desirable. \textit{Association rules} are an existing approach, and association rule mining in FCA has been discussed \cite{ganter2016conceptual,lakhal2005efficient}. Association rules, however, use \textit{confidence} and \textit{support} as a means of discovering relationships. These relationships then do not correspond to some formal pattern of reasoning—e.g., rational consequence relations—but are rather a flavour of `majority rules'.

